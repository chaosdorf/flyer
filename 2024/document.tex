\documentclass{article}
\usepackage{tabularx}
\usepackage{url}
\usepackage{svg}
\svgsetup{inkscapeversion=1}
\usepackage{fontspec}
\newfontfamily\computer{Computerfont_1994 Regular}[
    Extension=.ttf,
    Path=../design/typography/computerfont/,
]
\setmainfont{CPMono_v07 Plain}[
    Extension=.otf,
    Path=../design/typography/cp-mono/,
]
\usepackage{geometry}
 \geometry{
 a6paper,
 left=2.5mm,
 top=10mm,
 bottom=2.5mm,
 }
\begin{document}
    \begin{center}
        \thispagestyle{empty}
        \hspace*{1.5em}
        \includesvg[scale=0.5]{../design/logos/dorfhoernchen-monochrome.svg}

        \vspace*{5em}
        \hspace*{3em}
        \includesvg[scale=0.25]{../design/logos/logo-monochrome.svg}

        \vfill

        \begingroup
            \setlength{\tabcolsep}{10pt} % Default value: 6pt
            \renewcommand{\arraystretch}{2} % Default value: 1

            \begin{tabularx}{\textwidth}{ X X }
                Sonnenstraße 58, \newline 40227 Düsseldorf & \vfill https://chaosdorf.de/  \\
                mail@chaosdorf.de & @chaosdorf@chaos.social \\ 
                +4921174958156 & \#chaosdorf:matrix.org
            \end{tabularx}
        \endgroup
    \end{center}

    \newgeometry{
        a6paper,
        left=15mm,
        top=5mm,
        bottom=5mm,
        right=15mm,
    }
    \newpage
    \thispagestyle{empty}

    \section*{Spaß am Gerät!}
    Das Chaosdorf ist ein Ort zum Lernen, Ausprobieren und Erfinden - ein
    Hackspace für dich und für alle, die gemeinsam weiter kommen möchten.
    Hier gibt es Platz für deine Ideen und die nötige Inspiration,
    diese zu verwirklichen.

    \vfill

    \section*{Freitagsfoo}
    Unser offener Abend ohne spezifisches Thema;
    oft mit spontanen Kurzvorträgen ab 21:00 Uhr

    \begin{tabular}{ c }
        \\
        \hline
        Jeden Freitag ab 18:00 Uhr \\
        \hline
    \end{tabular}

    \section*{Pythonfoo}
    Wir versuchen uns mit mehr oder weniger nützlichen Projekten Python und
    die Programmierung an sich gegenseitig beizubringen.

    \begin{tabular}{ c }
        \\
        \hline
        Jeden Donnerstag ab 18:00 Uhr \\
        \hline
    \end{tabular}

    \section*{mehr?}
    https://chaosdorf.de/termine
\end{document}
